\chapter{Limits}
\lecturedetails{14 November 2017}{M Fiore, N Licker, S Borgeaud}

\section{Products}

In a category with products, we have a $\prod_{i \in I} X_i$ such that for
all $k \in I$:

\begin{center}
  \begin{tikzcd}
    \prod_{i \in I} X_i
      \arrow["\pi_k"]{d}
    \\
    X_k
  \end{tikzcd}
\end{center}

\section{Equalizer}

An equalizer is an object $E$ and a morphism $e: E \to A$ such that $fe=ge$ and
for all objects $X$ and morphisms $x: X \to A$ satisfying $fx=gx$, there is a
unique $u: X \to E$ such that $eu = x$.

\begin{center}
  \begin{tikzcd}
    E
      \arrow["e"]{r}
    &
    A
      \arrow["f"above, shift left=+1]{r}
      \arrow["g"below, shift left=-1]{r}
    &
    B
    \\
    &
    \forall X
      \arrow["x"]{u}
      \arrow["\exists! u", dotted]{ul}
    &
  \end{tikzcd}
\end{center}

\begin{example}

In $\catset$, $E = \Big\{ a \in A \mid f(a) = g(a)\Big\}$ is an equalizer
for $A$, $f$ and $g$.

\end{example}

\section{Pullbacks}

Pullbacks are limits for the diagram:

\begin{center}
  \begin{tikzcd}
    &
    B
      \arrow["g"]{d}
    \\
    A
      \arrow["f"]{r}
    &
    C
  \end{tikzcd}
\end{center}


$P$, $p:P\to A$ and $q:P\to $B is a pullback of $f:A \to C$ and $g:B \to C$ if
for all objects $X$ and morphisms $u:X \to A$ and $v:X \to B$ such that $fu=fv$,
there is a unique $w:X \to P$ such that $pw=u$ and $qw=v$:

\begin{center}
  \begin{tikzcd}
    X
      \arrow["\exists!w", dotted]{dr}
      \arrow["v", bend left=20]{drr}
      \arrow["u", bend right=20]{ddr}
    &
    &
    \\
    &
    P
      \arrow["q"]{r}
      \arrow["p"]{d}
      \arrow[phantom, "\lrcorner", pos=0]{dr}
    &
    B
      \arrow["g"]{d}
    \\
    &
    A
      \arrow["f"]{r}
    &
    C
  \end{tikzcd}
\end{center}

\begin{example}
In $\catset$, $P = \Big\{(a, b) \in A \times B \mid f(a) = g(b)\Big\}$ is a
pullback for $A$, $f$ and $g$.
\end{example}

\begin{definition}
A diagram D of small cagories $\mathbb{G}$ in a category $\mathcal{C}$ is
a functor $\mathbb{G} \xrightarrow{D} \mathcal{C}$.
\end{definition}

\begin{example}
If $\mathbb{G}$ is discrete (has only identity arrows), then we have families
$D_m, m \in \mathbb{G}$ that are diagrams for products.
\end{example}

\begin{example}
Diagrams for equalisers

$$
G =
\begin{tikzcd}
  \bullet
    \arrow[shift left=+1]{r}
    \arrow[shift left=-1]{r}
  &
  \bullet
\end{tikzcd}
$$

\end{example}

\begin{example}
Diagrams for pullbacks
  \begin{center}
    \begin{tikzcd}
      &
      \bullet
        \arrow[]{d}
      \\
      \bullet
        \arrow[]{r}
      &
      \bullet
    \end{tikzcd}
  \end{center}
\end{example}

\begin{definition}

A cone for a diagram $D:\mathbb{G} \to \mathbb{G}$ consists of:

\begin{itemize}
\item an object $x \in \mathcal{C}$
\item together with a family $\chi: X \to D m, m \in \mathbb{G}$

Such that for all $m, n \in \mathbb{G}$ and $m \xrightarrow{e} n$:

\begin{center}
  \begin{tikzcd}
    D m
      \arrow["D e"]{rr}
    &
    &
    D n
    \\&&\\
    &
    X
      \arrow["\chi_m"]{uul}
      \arrow["\chi_n"]{uur}
    &
  \end{tikzcd}
\end{center}

\end{itemize}

A limit of $D: \mathbb{G} \to \mathcal{C}$ is a terminal cone, that is:

$$
\Big(L, \lambda = \big\{ \lambda_m: L \to D m \big\}_{m \in \mathbb{G}}\Big)
$$

such that for all
$\Big(X, \chi = \big\{ \chi_m: X \to D m \big\}_{m \ in \mathbb{G}}\Big)$ and
$m \in \mathbb{G}$, there is a unique $u$ making the following diagram commute:

\begin{center}
  \begin{tikzcd}
    X
      \arrow["\exists!u"]{rr}
      \arrow["\chi_m"below]{rdd}
    &
    &
    L
      \arrow["\lambda_m"]{ldd}
    \\&&\\
    &
    D m
    &
  \end{tikzcd}
\end{center}

\end{definition}

\begin{proposition}
A category with small products and equalizers has all small limits.

Idea: for a $D:\mathbb{G} \to \mathcal{C}$.

\begin{center}
  \begin{tikzcd}
    \prod_{k\in\mathbb{G}}{D k}
      \arrow["\pi_n"]{rr}
      \arrow["\pi_m"]{rd}
    &
    &
    D n
    \\
    &
    D m
      \arrow["D e"]{ur}
    &
  \end{tikzcd}
\end{center}

\begin{center}
  \begin{tikzcd}
    L
      \arrow["eq"]{r}
    &
    \prod_{k\in\mathbb{G}}{D k}
      \arrow
        [
          "{\langle D e \cdot \pi_m\rangle_{e:m\to n}}"above,
          shift left=+1
        ]
        {rrr}
      \arrow
        [
          "{\langle \pi_n \rangle_{e:m\to n}}"below,
          shift left=-1
        ]
        {rrr}
    &
    &
    &
    \prod_{(e:m\to n) \in \mathbb{G}}
    D n
  \end{tikzcd}
\end{center}

The limit is L together with

\begin{center}
  \begin{tikzcd}
    (L
      \arrow["eq"]{r}
      \arrow["\lambda_i"below, bend right=20]{rr}
    &
    \prod_k D k
      \arrow["\pi_i"]{r}
    &
    D i )_{i \in \mathbb{G}}
  \end{tikzcd}
\end{center}

\end{proposition}

\begin{lemma}

If a category has pullbacks and a terminal object, it has finite products.

\begin{center}
  \begin{tikzcd}
    P
      \arrow[phantom, "\lrcorner", pos=0]{dr}
      \arrow[]{r}
      \arrow[]{d}
    &
    B
      \arrow[]{d}
    \\
    A
      \arrow[]{r}
    &
    1
  \end{tikzcd}
  \begin{tikzcd}
    &
    P
      \arrow[""]{dl}
      \arrow[""]{dr}
    &
    \\
    A
    &
    &
    B
  \end{tikzcd}
\end{center}

\end{lemma}

\begin{lemma}

If a category has pullbacks and a terminal object, it has equalizers.

\begin{center}
  \begin{tikzcd}
    A
      \arrow["f"above, shift left=+1]{r}
      \arrow["g"below, shift left=-1]{r}
    &
    B
  \end{tikzcd}
\end{center}

\begin{center}
  \begin{tikzcd}
    E
      \arrow[]{r}
      \arrow[]{d}
    &
    A
      \arrow["{\langle id_A, g \rangle}"]{d}
    \\
    A
      \arrow["{\langle id_A, g \rangle}"below]{r}
    &
    A \times B
  \end{tikzcd}
\end{center}

Intuitively, in $\catset$:
$$
  E
  =
  \big\{(a, a') \mid (a, f(a) = (a', g(a')))\big\}
  =
  \big\{(a, a') \mid a = a' \wedge f(a) = g(a')\big\}
  \cong
  \big\{a \mid f(a) = g(a)\big\}
$$

\end{lemma}

\begin{exercise}
Construct the equalizer $E$
\end{exercise}

\section{Pullback Lemma}
\begin{theorem}[Pullback Lemma]
Suppose the following diagram commutes in a category $\mathcal{C}$
\begin{center}
\begin{tikzcd}
F \arrow[d, "h"'] \arrow[r, "f'"] & E \arrow[phantom, "\lrcorner", pos=0]{dr} \arrow[r, "g'"] \arrow[d, "h'"'] & D \arrow[d, "h''"'] \\
A \arrow[r, "f"'] & B \arrow[r, "g"'] & C
\end{tikzcd}
\end{center}
And suppose the right square is a pullback.

Then, the left square is a pullback if and only iff the outer rectangle is a pullback.
\end{theorem}
\begin{exercise}
	Proof the Pullback Lemma.
\end{exercise}

\subsection{Monomorphisms are pullbacks}

\begin{proposition}
A morphism $m: A \to B$ is a monomorphism iff 

\begin{center}
\begin{tikzcd}
A \arrow[r, "id_A"] \arrow[d, "id_A"'] \arrow[phantom, "\lrcorner", pos=0]{dr} & A \arrow[d] \\
A \arrow[r, "m"'] & B
\end{tikzcd}	
\end{center}
is a pullback
\end{proposition}

\begin{exercise}
	Prove the proposition.
\end{exercise}

\begin{proposition}
In a pullback
\begin{center}
\begin{tikzcd}
P \arrow[r] \arrow[d, "p"'] \arrow[phantom, "\lrcorner", pos=0]{dr}  & S \arrow[d, "m"] \\
A \arrow[r] & B
\end{tikzcd}
\end{center}
if $m: S \to B$ is a monomorphism, then $p: P \to A$ is also a monomorphism.
\end{proposition}
Intuitively, monomorphisms are like injections and therefore are like predicates, as they select a subset of the elements in the range of the function. In the category $\catset$, we have the pullback 
\begin{center}
\begin{tikzcd}
\{ (a,s) \mid f(a) = m(s) \}  \arrow[r] \arrow[d] \arrow[phantom, "\lrcorner", pos=0]{dr}  & S \arrow[d, "m", tail] \\
A \arrow[r, "f"'] & B
\end{tikzcd}
\end{center}
But
\[ \{ (a,s) \mid f(a) = m(s) \} \]
is in bijection with
\[ \{ a \mid f(a) \text{ in the image of } S \text{ under } m \} \]
which is the inverse image of $S$
\[ f^{-1}(m[S])\]

The idea is therefore given intuitively by the pullback
\begin{center}
\begin{tikzcd}
f^{-1}(S) \arrow[r] \arrow[d, tail] \arrow[phantom, "\lrcorner", pos=0]{dr}  & S \arrow[d, "m", tail] \\
A \arrow[r, "f"'] & B
\end{tikzcd}
\end{center}

\begin{remark}
For a morphism $f: A \to B$ we have the pullback
\begin{center}
\begin{tikzcd}
A \arrow[r, "f"] \arrow[d, "id_A"]  \arrow[phantom, "\lrcorner", pos=0]{dr} & B \arrow[d, "id_B"] \\
A \arrow[r, "f"'] & B
\end{tikzcd}
\end{center}
This corresponds to the fact that $f^{-1}(B) = A$.
\end{remark}

\begin{remark}
We also have that
\begin{center}
\begin{tikzcd}
f^{-1}(g^{-1}(S)) \arrow[r] \arrow[d] & g^{-1}(S) \arrow[r] \arrow[d] & S \arrow[d, tail] \\
A \arrow[r, "f"'] & B \arrow[r,"g"] & C	
\end{tikzcd}
\end{center}
corresponds to 
\[ f^{-1}(g^{-1}(S)) = (g \circ f)^{-1} \]
This fact corresponds to the first half of the pullback lemma.
\end{remark}

\section{Subobjects}
\begin{definition}[$\approx$]
	Given two monomorphisms $m_1: S_1 \to A$ and $m_2: S_2 \to A$, i.e.
	\begin{center}
		\begin{tikzcd}
S_1 \arrow[rd, "m_1"', tail] &  & S_2 \arrow[ld, "m_2", tail] \\
 & A & 
\end{tikzcd}
	\end{center}
\end{definition}
we say $m_1 \approx m_2$ holds iff there exists an isomorphism 
\begin{tikzcd}
S_1 \arrow[r, "\cong"] & S_2
\end{tikzcd} such that
\begin{center}
\begin{tikzcd}
S_1 \arrow[rd, "m_1"', tail] \arrow[rr, "\cong"] &  & S_2 \arrow[ld, "m_2", tail] \\
 & A & 
\end{tikzcd}
\end{center}
commutes.

\begin{definition}[Suboject]
	A suboject of an object $A$ is an equivalence class $[m]_{\approx}$ for $m$ a monomorphism into $A$.
\end{definition}

\begin{definition}[$\mathit{Sub}$ functor]
	For a category $\mathcal{C}$ with pullbacks we have a functor $\mathit{Sub}: \mathcal{C}^\op \to  \catset$ mapping objects
	
\begin{center}
\begin{tikzcd}
X \arrow[rr, maps to] &  & \mathit{Sub}(X) = \{ [m]_{\approx} \mid m \text{ a monomorphism into } X\} &  \\
\end{tikzcd}
\end{center}
and mapping morphisms $f: Y \to X$ to functions $\mathit{Sub}(f): \mathit{Sub}(X) \to \mathit{Sub}(Y)$ given by
\begin{center}
\begin{tikzcd}
\lbrack m \rbrack_{\approx} \arrow[r, maps to] & \lbrack f^*(m)\rbrack_{\approx}
\end{tikzcd}
\end{center}
where $f^*(m)$ is given by the pullback
\begin{center}
\begin{tikzcd}
f^*(S) \arrow[r] \arrow[d, "f^*(m)"'] \arrow[phantom, "\lrcorner", pos=0]{dr} & S \arrow[d, "m", tail] \\
Y \arrow[r, "f"'] & X
\end{tikzcd}
\end{center}
\end{definition}
\begin{remark}
Functoriality follows from the pullback lemma.	
\end{remark}

\begin{exercise}
What is the representation for $\mathit{Sub}$?
\begin{center}
\begin{tikzcd}
\mathcal{C}(\_,\Omega) \arrow[r, "\cong", Rightarrow] & \mathit{Sub}(\_)
\end{tikzcd}
\end{center}	
\end{exercise}
 