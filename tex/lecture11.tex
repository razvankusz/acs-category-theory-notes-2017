\newcommand*{\threesim}{%
\mathrel{\vcenter{\offinterlineskip
\hbox{$\sim$}\vskip-.45ex\hbox{$\sim$}\vskip-.45ex\hbox{$\sim$}}}}

\chapter{Colimits}
\lecturedetails{9 November 2017}{M Fiore, S Lau, D Szamozvancev}

We can generalize products and coproducts, amongst other similar notions in
category theory, using the idea of limits and colimits.  We begin with a
motivating example.

\section{Colimits of $\omega$-chains}

\subsection{Lubs of $\omega$-chains}

Consider a preorder $(P, \leq)$ regarded as a category  $\lscat{P}$.

In $\lscat{P}$,  products are meets ($\wedge, \bigwedge$) and coproducts are
joins ($\vee,\bigvee$). These constructions were previously defined for pairs
and, more generally, for subsets of elements.

In domain theory, we are interested in other kinds of least upper bounds. In
particular, we are interested in least upper bounds of $\omega$-chains (or
countably increasing chains)
\[
    x_0 \leq x_1 \leq x_2 \leq \cdots \leq x_n \leq \cdots \quad (n \in \nats)
\]

The least upper bound of an $\omega$-chain $\seq{ x_n }_{n\in\nats}$, if it
exists, is denoted $\lub_{n \in \nats}x_n$ and in $\lscat{P}$ has the
universal property of being an upper bound of the $\omega$-chain that is least
amongst all upper bounds of the $\omega$-chain.

To generalize these kinds of constructions in a category, we need something
analogous to $\omega$-chains and to the universal definition of least upper
bounds.

\subsection{$\omega$-chains}

\begin{definition}[$\omega$-chain]
Informally, an $\omega$-chain in a category is given by a diagram as follows
\[
    X_0 \morpharrow X_1 \morpharrow X_2 \morpharrow \cdots \morpharrow X_i
    \morpharrow \cdots
    \qquad(i\in\nats)
\]
that is, a family of objects $X_i$~($i\in\nats$) with a map from $X_i$ to
$X_{i+1}$ for each $i \in \nats$.

Formally, consider the category $\underline{\omega}$ with objects the natural
numbers and arrows $m \morpharrow n$ iff $m \leq n$. An $\omega$-chain in a
category $\lscat{C}$ is then a functor $X : \underline{\omega} \to \lscat{C}$
such that
\begin{align*}
    n & \mapsto X_n & (\text{object map}) \\
    (m \leq n) & \mapsto (X_m \morpharrow X_n) &(\text{morphism map})
\end{align*}

\end{definition}

\begin{definition}[Colimits of $\omega$-chains]
The colimit of an $\omega$-chain
\[
    X_0 \morpharrow X_1 \morpharrow X_2 \morpharrow \cdots \morpharrow X_i
    \morpharrow \cdots
    \qquad(i\in\nats)
\]
say
\begin{center}
\begin{tikzcd}[row sep=2cm, column sep=2cm]
  X_0    \arrow[r]
         \arrow[drr,"\gamma_0",swap]
& X_1    \arrow[r]
         \arrow[dr,"\gamma_1",swap]
& \cdots \arrow[r]
         \arrow[d,dashed]
& X_n    \arrow[r]
         \arrow[dl, "\gamma_n"]
& \cdots \arrow[dll, dashed]
\\
&& C
\end{tikzcd}
\end{center}
consists of a object $C$ (called a \emph{vertex}) together with a family of
arrows $\langle \gamma_n : X_n \morpharrow C \rangle _{n \in \nats}$ (called a
\emph{cocone}, with dual \emph{cone} for limits) such that the following
diagram commutes 
\begin{center}
\begin{tikzcd}[column sep=small]
   X_n     \arrow[rr]
           \arrow[dr, "\gamma_n", swap]
&& X_{n+1} \arrow[dl, "\gamma_{n+1}"]
\\
& C
\end{tikzcd}
\qquad$(n \in \nats)$
\end{center}
and is initial universal, in the sense that for any other cocone, say given by
an object $D$ and family of arrows $\langle \delta_n : X_n \rightarrow
D\rangle_{n \in \nats}$ such that the following diagram commutes,
\begin{center}
\begin{tikzcd}[column sep=small]
X_n   \arrow[rr]
      \arrow[dr, "\delta_n", swap]
&& X_{n+1}
      \arrow[dl, "\delta_{n+1}"]
\\
& D
\end{tikzcd}
\end{center}
there exists a unique map
\[
    C \morpharr{u} D
\]
such that the following commutes 
\begin{center}
\begin{tikzcd}[column sep=small]
&  X_n \arrow[dl, "\gamma_n",swap]
       \arrow[dr, "\delta_n"]
\\
   C  \arrow [rr, "\exists!\,u",swap]
&& D
\end{tikzcd}
\end{center}
for all $n \in \nats$.
\end{definition}

\begin{remark}
Intuitively, this universal property generalizes the notion of least upper
bound from preorders to categories. For $C$ is an ``upper bound'' (witnessed
by $\gamma$) and if $D$ is also an ``upper bound'' (witnessed by $\delta$),
then there is a unique arrow from $C$ to $D$, expressing ``$C \leq D$'', and
mapping the former to the latter.
\end{remark}

This idea of a colimit generalizes to other diagrams. For instance, consider the
following expressing countable sums (the colimit is denoted by
$\coprod_{k\in\nats} Y_k$):
\begin{center}
\begin{tikzcd}[column sep=small]
& 
Y_n 
\arrow[dl, "\iota_n",swap] 
\arrow[dr, "\delta_n"]
\\
\coprod_{k \in \nats} Y_k
  \arrow [rr, "{\exists !\, [\delta_k]_{k\in\nats}}",swap, dashed]
& & 
D
\end{tikzcd}
\qquad($n \in \nats$)
\end{center}

N.B.\enspace Here $Y_n$~($n\in\nats$) is discrete, whereas the $\omega$-chain
used above in defining $\omega$-colimits had structure. We can define limits
and colimits for arbitrary diagrams, though they may not necessarily exist.

\section{Coequalisers}

As another useful example of a colimit, we consider the diagram called a \emph{parallel pair}, consisting of two objects in a category $\lscat C$ with two parallel morphisms between them:

\begin{center}
\begin{tikzcd}
    A \arrow[r, shift left=5pt, "f"]
      \arrow[r,shift right=5pt, swap, "g"]
    & B
\end{tikzcd}
\end{center}

\begin{definition}[Coequaliser]
A \emph{coequaliser} is the colimit of a parallel pair diagram.
\end{definition}

As before, to define the coequaliser, we start by finding a suitable cocone for
this diagram:

\begin{center}
\begin{tikzcd}[column sep=1.5em, row sep=2.5em]
    A \arrow[rr, shift left=5pt, "f"]
      \arrow[rr, shift right=5pt, swap, "g"]
      \arrow[rd, "\alpha", swap]
    & {}
    & B \arrow[ld, "\beta"]
    \\
    {} & C
\end{tikzcd}
\end{center}

We want this diagram to commute \emph{serially}~(\ie~for each arrow in turn):
$\beta \comp f = \alpha$ and $\beta \comp g = \alpha$. In fact, we don't need
to provide $\alpha$ as it can just be defined in terms of $\beta$. All we need
is $\beta : B \to C$ such that $\beta \comp f = \beta \comp g$, that is, a
function which \emph{coequalises} $f$ and $g$.  The universal initial solution
to this problem is the \emph{coequaliser} of $f$ and $g$: for all 
$\zeta : B \to Z$ which coequalise $f$ and $g$, there exists a unique arrow 
$u : Z \to C$ such that the following diagram commutes:
\begin{center}
\begin{tikzcd}
      A \arrow[r, shift left=5pt, "f"]
        \arrow[r,shift right=5pt, swap, "g"]
    & B \arrow[r, "\beta"]
        \arrow[rd, swap, "\forall \zeta"]
    & C \arrow[d, dashed, "\exists! u"]
    \\
    {} & {} & \forall Z
\end{tikzcd}
\end{center}

\subsection{Coequalisers in sets}
Let's consider coequalisers in the familiar category of sets.

\begin{center}
\begin{tikzcd}
      S \arrow[r, shift left=5pt, "f"]
        \arrow[r, shift right=5pt, swap, "g"]
    & T \arrow[r, "\kappa"]
    & K
\end{tikzcd}
\end{center}

Given two sets $S$ and $T$, with two functions $f, g : S \to T$, the
coequaliser $\kappa : T \to K$ of $S$ and $T$ is a function such that 
$\kappa \comp f = \kappa \comp g$, that is,

\begin{equation*}
    \forall s \in S.\ \kappa (f(s)) = \kappa (g(s))
\end{equation*}

We want to find a way of forcing $f(s)$ to equal $g(s)$. Looking closer at the
set $T$, we would like to identify $f(s)$ and $g(s)$ for all $s \in S$.

\begin{center}
\begin{tikzpicture}
    [
        group/.style={ellipse, draw, minimum height=20pt,
                      minimum width=50pt, label=right:#1},
        subgroup/.style={rounded corners=10, draw, dashed},
        dot/.style={circle, fill, minimum width=2.5pt, inner sep=0pt}
    ]
    \node (fs) {$f(s)$};
    \node (gs) [below=10pt of fs] {$g(s)$};
    \node (fsp) [right=10pt of fs] {$f(s')$};
    \node (gsp) [below=10pt of fsp] {$g(s')$};
    \node (fspp) [below=10pt of gs] {$f(s'')$};
    \node (gspp) [right=10pt of fspp] {$g(s'')$};
    \node (s) [fit=(fs) (gs), subgroup] {};
    \node (sp) [fit=(fsp) (gsp), subgroup] {};
    \node (spp) [fit=(fspp) (gspp), subgroup] {};
    \node [fit=(s) (sp) (spp), group=T] {};
\end{tikzpicture}
\end{center}

Moreover, if, for example, $f(s)$ was equal to $g(s')$, we would like all of
$f(s)$, $g(s)$, $f(s')$ and $g(s')$ to be equal -- or, at least, equivalent.
This suggests defining an equivalence relation ensuring that the above elements
all reside in the same equivalence class. The function $\kappa$ would then map
every element $t \in T$ to its corresponding equivalence class in $K$, and
therefore $K$ would be the \emph{quotient set} of $T$ by the equivalence
relation.

Concretely, define the relation $ {}_{f}\hspace{-3px}\sim_{g}\ \subseteq T
\times T$ as:

\begin{equation*}
    {}_{f}\hspace{-3px}\sim_{g}\ 
    \eqdef\ 
    \bigsetof{\,\big(f(s), g(s)\big) \suchthat s \in S\,}
\end{equation*}
and let ${}_{f}\hspace{-3px}\approx_{g}$ be the least equivalence relation
that contains ${}_{f}\hspace{-3px}\sim_{g}$. That is, 
$x\: {}_f\hspace*{-3px}\approx_g y$ iff there is a finite sequence of elements
$z_0, z_1, \ldots, z_n$~$(n\in\nats)$ such that 
\[
 x = z_0\ 
 {}_f\hspace*{-3px}\sim_g 
 z_1 \
 {}_f\hspace*{-3px}\sim_g 
 \ \cdots \
 {}_f\hspace*{-3px}\sim_g 
 z_{n-1}\
 {}_f\hspace*{-3px}\sim_g 
 z_n = y
 \enspace.  
\]
Now, we define the coequaliser $\kappa$ of $f$ and $g$ as a function from $T$
to the quotient set $T_{\mbox{/${}_f\hspace*{-3px}\approx_g$}}$, the set of
equivalence classes of $T$ under ${}_f\hspace*{-3px}\approx_g$:

\begin{center}
\begin{tikzcd}[row sep=0.2em]
      S \arrow[r, shift left=5pt, "f"]
        \arrow[r, shift right=5pt, swap, "g"]
    & T \arrow[r, "\kappa"]
    & T_{\mbox{/${}_f\hspace*{-3px}\approx_g$}}
    \\
    {}
    & t \arrow[r, mapsto]
    & \left[ t \right]_{ {}_f\hspace*{-0px}\approx_g }
\end{tikzcd}
\end{center}

\begin{exercise}
    Show that this definition has the universal property of coequalisers.
\end{exercise}

\section{Pushouts}

Consider the following diagram, called a \emph{span}:

\begin{center}
\begin{tikzcd}
    C \arrow[r, "g"] \arrow[d, "f", swap]
    & B
    \\
    A
\end{tikzcd}
\end{center}

To define its colimit, called a \emph{pushout}, we again consider the required
commuting cocones:

\begin{center}
\begin{tikzcd}
    C \arrow[r, "g"] \arrow[d, "f", swap] \arrow[rd, "\gamma"]
    & B \arrow[d, "\beta"]
    \\
    A \arrow[r, "\alpha"]
    & K
\end{tikzcd}
\end{center}

We require this diagram to commute, 
\ie~$\alpha \comp f = \beta \comp g = \gamma$; again, $\gamma$ can be defined
from $\alpha$ or $\beta$ so it is usually omitted).  We also require the
pushout to be the universal initial solution: for all other cocones (also
called \emph{cospans}) $(u: A\to X\leftarrow B: v)$, there is a unique map
from $K$ to $X$ such that the following diagram commutes:
\begin{center}
\begin{tikzcd}
    C \rar{g} \dar[swap]{f}
    & B \arrow[d, "\beta"] \arrow[ddr, bend left=30, "\forall v"] &[-0.5em] {}
    \\
    A \arrow[r, "\alpha"] \arrow[rrd, bend right=30, "\forall u"]
    & K \arrow[rd, dashed, "\exists! w"] & {}
    \\[-0.5em]
    && \forall X
\end{tikzcd}
\end{center}

The following notation is used for pushouts:
\begin{center}
\begin{tikzcd}
    C \rar{g} \dar{f} \arrow[dr, phantom, "\ulcorner" font=\Large, very near end]
    & B \dar{\beta}
    \\
    A \rar[swap]{\alpha}
    & K
\end{tikzcd}
\end{center}

\paragraph{Pushouts from coproducts and coequalisers.}

The above looks similar to the coproduct diagram, so our initial guess to
construct a pushout could be to take $A + B$:

\begin{center}
\begin{tikzcd}
    C \rar{g} \dar[swap]{f}
    & B \arrow[d, "\iota_2", dotted]
    \\
    A \arrow[swap,r, "\iota_1", dotted]
    & A + B
\end{tikzcd}
\end{center}

However, this diagram does not necessarily commute: there is no reason for
which $\iota_1 \comp f = \iota_2 \comp g$ in general.  The trick is then to
universally force this commutativity by means of the coequaliser 
$c = \text{coeq} (\iota_1 \comp f,\ \iota_2 \comp g )$:

\begin{center}
\begin{tikzcd}
    C \rar{g} \dar[swap]{f}
    & B \arrow[d, "\iota_2", dotted] \arrow[ddr, bend left=30, "\beta"]
    &[-0.5em] {}
    \\
    A \arrow[r, "\iota_1", dotted] \arrow[drr, bend right=30, "\alpha"]
    & A + B \arrow[rd, "c"] & {}
    \\[-0.5em]
    && K
\end{tikzcd}
\end{center}

\exercise{ 
  Show that the above construction yields a pushout.
}

\begin{example}
In $\catset$, the pushout is then a quotient set (by the appropriate
equivalence relation) of the disjoint union of $A$ and $B$.
\begin{center}
\begin{tikzcd}
    C \rar \dar \arrow[dr, phantom, "\ulcorner" font=\Large, very near end]
    & B \dar
    \\
    A \rar
    & (A \uplus B)_{_\mbox{\hspace*{-1px}/$\approx$}}
\end{tikzcd}
\end{center}
\end{example}

\section{Finite colimits}

So far we looked at special cases of colimits for $\omega$-chains, countable
sums, coequalisers and pushouts. However, the construction we saw in the
previous section can be applied to any finite diagram to find its colimit.

\begin{proposition}
If a category has initial object, binary sums, and coequalisers, then all
finite diagrams have a colimit.
\end{proposition}

In fact, we can state a more general result:

\begin{proposition}
If a category has all small coproducts and coequalisers, all small diagrams
have a colimit.
\end{proposition}

To demonstrate this, consider an arbitrary diagram such as this one:

\begin{center}
\begin{tikzcd}
C \arrow[r,"g"] \arrow[d,"f",swap] & B \arrow[r,"j"] \arrow[d,"i"] & E \\
A & D &
\end{tikzcd}
\end{center}

The ``algorithm'' for finding its colimit is simply a repetition of the
construction we performed for pushouts.  First we consider the sum
$K=A+B+C+D+E$.  To make the family 
\begin{center}\begin{tikzcd}[column sep=3em, row sep=2.5em]
  A 
  \arrow[swap,drr,"\iota_A"] 
  & B 
  \arrow[dr,"\iota_B"] 
  & C 
  \arrow[d,"\iota_C"] 
  & 
  D 
  \arrow[dl,"\iota_D",swap] 
  & E 
  \arrow[dll,"\iota_E"] 
  \\
  && K &&&
\end{tikzcd}\end{center}
into a cocone for the diagram, we repeatedly enforce compatibility with each
of the arrows in the diagram by means of coequalisers.
\begin{itemize}
\item
  To enforce compatibility with $f: C\to A$, we take the coequaliser
  \begin{center}\begin{tikzcd}[row sep=0em]
    C 
    \arrow[rd,"f",swap]
    \arrow[rr,shift left=10pt,"\iota_C"]
    & 
    & K \arrow[r,two heads,"\kappa_f"] & K_f
    \\
    & A \arrow[ru,"\iota_A",swap] & 
  \end{tikzcd}\end{center}

\item
  To further enforce compatibility with $g:C\to B$, we take the coequaliser
  \begin{center}\begin{tikzcd}[row sep=0em]
    C 
    \arrow[rd,"g",swap]
    \arrow[rr,shift left=10pt,"\kappa_f\comp\iota_C"]
    & 
    & K_f \arrow[r,two heads,"\kappa_{g}"] & K_{f,g}
    \\
    & B \arrow[ru,"\kappa_f\comp\iota_B",swap] & 
  \end{tikzcd}\end{center}

\item
  To further enforce compatibility with $i:B\to D$, we take the coequaliser
  \begin{center}\begin{tikzcd}[row sep=0em,col sep=3em]
    B 
    \arrow[rd,"i",swap]
    \arrow[rr,shift left=10pt,"\kappa_g\comp\kappa_f\comp\iota_B"]
    & 
    & K_{f,g} \arrow[r,two heads,"\kappa_i"] & K_{f,g,i}
    \\
    & D \arrow[ru,swap,"\kappa_g\comp\kappa_f\comp\iota_D"] & 
  \end{tikzcd}\end{center}

\item
  Finally, to further enforce compatibility with $j:B\to E$, we take the
  coequaliser 
  \begin{center}\begin{tikzcd}[row sep=0em,col sep=3em]
    B 
    \arrow[rd,"j",swap]
    \arrow[rr,shift
    left=10pt,"\kappa_i\comp\kappa_g\comp\kappa_f\comp\iota_B"
    ]
    & 
    & K_{f,g,i} \arrow[r,two heads,"\kappa_j"] & K_{f,g,i,j}
    \\
    & E
    \arrow[ru,swap,"\kappa_i\comp\kappa_g\comp\kappa_f\comp\iota_E"] 
    & 
  \end{tikzcd}\end{center}
\end{itemize}
The family 
\[
  \bigsetof{\,
  \kappa_j\comp\kappa_i\comp\kappa_g\comp\kappa_f\comp\iota_X 
  : X\to K_{f,g,i,j}
  \,}_{X\in\setof{A,B,C,D,E}}
\]
is a colimiting cocone.

Alternatively, we could have done the above construction ``in parallel'' with
just one coequalizer:
\begin{center}\begin{tikzcd}
\dom(f)+\dom(g)+\dom(i)+\dom(j)
\arrow[rr,shift left=10pt,"{[\,\iota_A\comp f\,,\,\iota_B\comp
  g\,,\,\iota_D\comp i\,,\,\iota_E\comp j\,]}"]
\arrow[swap,rr,shift
right=10pt,"{[\,\iota_C\,,\,\iota_C\,,\,\iota_B\,,\,\iota_B\,]}"]
&& 
A+B+C+D+E
\arrow[r,two heads,"q"]
& 
Q
\end{tikzcd}\end{center}
obtaining the colimiting cocone
\[
  \bigsetof{\, q\comp\iota_X : X\to Q \,}_{X\in\setof{A,B,C,D,E}}
\]

