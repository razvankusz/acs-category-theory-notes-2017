\chapter{Natural isomorphisms}
\lecturedetails{2 November 2017}{M Fiore, A Ivašković, D Makwana }

%TODO: intro/clarification from start of the lecture?

We still focus our attention on natural transformations, and are moving
towards introducing \emph{representations}.

\section{Definition and basic properties}

\begin{definition}[Natural isomorphisms]
Given functors $F, G : \lscat A \to \lscat B$, a natural transformation
$\varphi: F \natto G$ is a \emph{natural isomorphism} whenever there exists a
(necessarily unique) natural transformation $\varphi^{-1}: G \natto F$ such
that: 
\[ \varphi \comp \varphi^{-1} = \id{G} \]
\[ \varphi^{-1} \comp \varphi = \id{F} \]
\end{definition}

\begin{theorem}
For functors $F, G : \lscat A \to \lscat B$, a natural transformation
$\varphi: F \natto G$ is a natural isomorphism if and only if:
\[ \forall\, A \in \lscat A.\ \text{$\varphi_A: FA \to GA$ is an isomorphism} \]
\end{theorem}

\begin{proof}
We need to prove both implications of this statement.

\begin{itemize}

\item[($\Rightarrow$)] Let $\varphi$ be an isomorphism. So there exists a
$\varphi^{-1}$ that satisfies the properties from the definition:
\[ \varphi \comp \varphi^{-1} = \id{G} \]
Let $A \in \lscat A$. Then, since
\[ (\varphi \comp \varphi^{-1})_A = \varphi_A \comp \varphi_A^{-1} \]
\[ ( \id{G} )_A = \id{GA} \]
we can conclude:
\[ \varphi_A \comp \varphi_A^{-1} = \id{GA} \]

Similarly:
\[ \varphi_A^{-1} \comp \varphi_A = (\varphi^{-1} \comp \varphi)_A = (\id{F})_A
= \id{FA} \]

\item[($\Leftarrow$)] Suppose $\varphi_A: FA \to GA$ has an inverse
$\varphi_A^{-1}: GA \to FA$ for all $A \in \lscat A$. Then define:
\[ \varphi^{-1} \eqdef \setof{\varphi_A^{-1}: GA \to FA}_{A\in\lscat A} \]
Then it is clear that:
\[ \varphi \comp \varphi^{-1} = \id{G} \]
\[ \varphi^{-1} \comp \varphi = \id{F} \]
It remains to show that the family $\varphi^{-1}$ is a natural transformation.
That is, we want to prove that the following diagram commutes:

\begin{center}
\begin{tikzcd}
    A \arrow{d}{\text{in $\lscat A$}}[swap]{\forall\,f} &
    GA \arrow{r}{\varphi_A^{-1}} \arrow{d}{Gf} & FA \arrow{d}{Ff} \\
    A & GA^{\prime} \arrow{r}{\varphi_{A^{\prime}}^{-1}} & FA^{\prime}
\end{tikzcd}
\end{center}

This is not difficult: since $\varphi$ is natural, the following diagram
commutes:
\begin{center}
\begin{tikzcd}
    FA \arrow{r}{\varphi_A} \arrow{d}{Ff} & GA \arrow{d}{Gf} \\
    FA^{\prime} \arrow{r}{\varphi_{A^{\prime}}} & GA^{\prime}
\end{tikzcd}
\end{center}
for all $f:A\to A'$ in $\lscat A$;
that is, $Gf \comp \varphi_A = \varphi_{A^{\prime}} \comp Ff$. Left composing
both sides of the equality with $\varphi_{A^{\prime}}^{-1}$ and right composing
them with $\varphi_{A}^{-1}$, we have
\[ \varphi_{A^{\prime}}^{-1} \comp Gf \comp \varphi_A \comp \varphi_A^{-1} =
\varphi_{A^{\prime}}^{-1} \comp \varphi_{A^{\prime}} \comp Ff
\comp \varphi_A^{-1} \]
which is just
\[ \varphi_{A^{\prime}}^{-1} \comp Gf = Ff \comp \varphi_A^{-1} \]
which is what we wanted to prove. Therefore $\varphi^{-1}$ is a natural
isomorphism.

\end{itemize}
\end{proof}

Intuitively, this means that, for a natural transformation to be an isomorphism,
it has to be an isomorphism at `every level' or `for every component'.

\begin{exercise}
Investigate whether the analogous result holds for natural monomorphisms,
epimorphisms, sections and retractions (which are all defined similarly to
natural isomorphisms).
\end{exercise}

\section{Representable functors}

Given a functor $F: \lscat A \times \lscat B \to \lscat C$ (so it is a functor
`of two variables'), define the functor $F (A, \phold): \lscat B \to \lscat C$
for a fixed $A$:

\begin{center}
\begin{tikzcd}
B \arrow[dd, "g"{name=L, right}] & &
F (A, B) \arrow[dd, "F(\id{A}{,} g)"{name=R, left}] \\
\arrow[rr, from=L, to=R, mapsto] & & \\
B^{\prime} &  & F (A, B^{\prime})
\end{tikzcd}
\end{center}

Intuitively, this means that we are performing a `restriction' of the first
argument.

Similarly, we may for a fixed $B \in \lscat B$ define the functor
$F (\phold, B) : \lscat A \to \lscat C$ as follows:

\begin{center}
\begin{tikzcd}
A \arrow[dd, "f"{name=L, right}] & &
    F (A, B) \arrow[dd, "F(f{,}\id{B})"{name=R, left}] \\
\arrow[rr, from=L, to=R, mapsto] & & \\
A^{\prime} &  & F (A^{\prime}, B)
\end{tikzcd}
\end{center}

We want to apply these to the hom functor:
\[ \lscat C (\phold, \Phold): {\lscat C}^\op \times \lscat C \to \catset \]
obtaining what is called a (contravariant) \emph{representable functor}: for
fixed $C \in \lscat C$, this is
\[
\lscat{C} (\phold, C): C^\op \to \catset 
\]
given by:

\begin{center}
\begin{tikzcd}
X & & \lscat C (X, C) \arrow[dd, ""{name=R, left}] & x \arrow[dd, mapsto] \\
& & & \\
Y \arrow[uu, "f"{name=L, right}] & & \lscat C (Y, C) & x \comp f
\arrow[rr, from=L, to=R, mapsto]
\end{tikzcd}
\end{center}

\section{Characterising natural isomorphisms between representable functors}

Now consider two fixed $A$ and $B$ in $\lscat C$. We are interested in the
natural transformations $\varphi$ of the form:
\[ 
  \lscat C(\phold, A) \natarr{\varphi} \lscat C(\phold, B) 
  : {\lscat C}^\op \to \catset 
\]
Is it possible to characterise them precisely? The answer is \emph{yes} --
this is because naturality is a very strong property.\footnote{Naturality for
  a family of morphisms is a polymorphism-like property of a program in a
  programming language. Think, for instance, of the polymorphic type
  $\forall\, \alpha.\ \alpha \to \alpha$ in pure ML or System~F: the only
  valid expression of this type is the identity function!} 
It turns out that this is not that difficult. Let $\varphi$ be the natural
family of functions:
\[ 
  \varphi 
  = 
  \setof{\varphi_X: \lscat C (X, A) \to \lscat C (X, B)}_{X \in \lscat C} 
\]
One of its members is $\varphi_A: \lscat C (A, A) \to \lscat C (A, B)$.
Define: 
\[ 
  \widehat{\varphi} 
  \,\eqdef\, 
  \varphi_A (\id{A}) : A \to B \text{ in } \lscat C 
\]
We claim that $\widehat{\varphi}$ completely determines all of $\varphi$ (that
is $\varphi_X$, for each $X\in\lscat C$) 
\emph{because of the naturality condition}.

This is seen from the following commutative diagram, for any $X$:

\begin{center}
\begin{tikzcd}
    A & \lscat C (A, A) \arrow[rrr, "\varphi_A"] \arrow[ddd, "\phold \comp f"]
    & & & \lscat C (A, B) \arrow[ddd, "\phold \comp f"] & \\
    & & \id{A} \arrow[lu, phantom, "\rotatebox{135}{$\in$}"] \arrow[rrr, mapsto]
    \arrow[ddd, mapsto]
    & & & \widehat{\varphi} \arrow[lu, phantom, "\rotatebox{135}{$\in$}"]
    \arrow[ddd, mapsto] \\
    & & & & & \\
    X \arrow[uuu, "f"] & \lscat C (X, A) \arrow[rrr, "\varphi_X"]
    & & & \lscat C (X, B) & \\
    & & \id{A} \comp f & & & \widehat{\varphi} \comp f \\
    & X \morpharr{f} A \arrow[rrr, mapsto]
    \arrow[uu, phantom, "\rotatebox{90}{$\in$}"]
    \arrow[ru, phantom, "\rotatebox{45}{$=$}"] & & &
    \varphi_X (f) \arrow[ru, phantom, "\rotatebox{45}{$=$}"]
    \arrow[uu, phantom, "\rotatebox{90}{$\in$}"] &
\end{tikzcd}
\end{center}

All the natural transformations 
$\lscat C(-,A)\Rightarrow\lscat C(-,B):\lscat C^\op\to\catset$ are
\textit{exactly} characterised by being in bijective correspondence with maps
$A \to B$ in $\lscat C$. Here we have only shown one direction of this claim.

\begin{exercise}
Show that every map $f: A \to B$ yields a natural transformation:
\[ \lscat C (\phold, A) \natarr{\widetilde{f}} \lscat C (\phold, B) \]
given by:
\[ \begin{array}{crcl}
    \widetilde{f}_X: & \lscat C (X, A) & \to & \lscat C (X, B) \\
    & x & \mapsto & f \comp x
\end{array} \]
\end{exercise}

\begin{exercise}
Show that the constructions $\widehat{(\phold)}$ and $\widetilde{(\phold)}$
are inverses of each other:
\[ \varphi \mapsto \widehat{\varphi} \mapsto \widetilde{\widehat{\varphi}} =
\varphi \]
\[ f \mapsto \widetilde{f} \mapsto \widehat{\widetilde{f}} = f \]
\end{exercise}

\subsection{A technique for proving isomorphisms}\label{subsec:proveiso}

Notation: a double-lined fraction is typically used to express a bijective
correspondence. With this notation, the result proven above, is sometimes
written as:

\[ 
  \Efrac
    {\lscat C (\phold, A) \natarrow \lscat C (\phold, B) }
    { \lscat C(A, B) }
\]

Suppose one was trying to show that two objects $A$ and $B$ are isomorphic.
This will usually involve constructing some isomorphism $ f : A \to B $.  In
practice, one can sometimes use the above result and instead come up with some
natural isomorphism $\lscat C (\phold, A) \natarr{\varphi} \lscat C (\phold,
B)$, which amounts to establishing 
\[
  \Efrac{X \to A}{X \to B} \quad\text{`natural in $X$'}
\]
and then define $f$ to be $\widehat{\varphi}$.

\section{A binary product construction}

Fix a diagram
\[
  A \overset{\alpha}{\longleftarrow} P \morpharr{\beta} B
\]
in a category $\lscat C$ and define the functor 
\[
  K_{A,B}(-) \eqdef \lscat C (\phold, A) \times \lscat C (\phold, B)
\]
(NB: The product of two functors is a functor.)

For every $X \in \lscat C^\op $, define the function 
\begin{center}
\begin{tikzcd}
\lscat C(X, P) \arrow[rr, "{(\alpha \comp \phold, \beta \comp \phold)}_X"]
& & \lscat C (X, A) \times \lscat C (X, B) 
\\
X \arrow[dd, "p", swap]
& & 
X \arrow[dd, "\alpha \comp p", end anchor={[xshift=1em]north west}, swap]
  \arrow[dd, "\beta \comp p", end anchor ={[xshift=-1em]north east}] 
\\
\arrow[rr, mapsto, start anchor={[xshift=2em]}, end anchor={[xshift=-4em]}] 
& & \  
\\
P & & A \hskip 4em B
\end{tikzcd}
\end{center}

We claim that the family of maps
\[
(\alpha \comp \phold, \beta \comp \phold) 
\ \eqdef \
\bigsetof{\,
  {(\alpha \comp \phold, \beta \comp \phold)}_X 
  :\lscat C(X,P) \to K_{A,B}(X)
  \,}_{X \in \lscat C^\op}
\]
satisifies the naturality condition:
\begin{center}
\begin{tikzcd}
&&&&
x \arrow[rr,mapsto]
&&
(\alpha\comp x,\beta\comp x)
&
\\
X 
&&&
p
\arrow[dd,mapsto]
& 
\lscat C(X,P) 
\arrow[rr, "{(\alpha \comp\;\phold,\;\beta \comp\;\phold)}_X"]
\arrow[dd, "{C(g,P)}"]
&& 
K (X) 
\arrow[dd, "K_{A,B}(g)"]
& (a,b) 
\arrow[dd, mapsto] 
\\
& & & & & 
\\
Y 
\arrow[uu, "\forall\, g"] 
&&
&
p\comp g
& 
\lscat C (Y, P) 
\arrow[rr, swap, "{(\alpha \comp\;\phold,\;\beta \comp\;\phold)}_Y"]
&& 
K_{A,B}(Y)
& 
(a \comp g, b \comp g)
\\
&&&&
y \arrow[rr,mapsto]
&&
(\alpha\comp y,\beta\comp y)
&
\\
\end{tikzcd}
\end{center}

Now, given an arbitrary natural transformation 
\[
  \varphi 
  : \lscat C(-,P) \Longrightarrow K_{A,B}(-)
  : \lscat C^\op \to \catset
\]
one can completely determine it by looking at its component $\varphi_P$ and
calculating what $id_P$ is mapped to:
\[
  \varphi_P 
  : \lscat C (P, P) \morpharrow \lscat C (P, A) \times \lscat C (P, B) 
\]
and 
\begin{equation}\label{Lecture9Exercise}
\begin{tabular}{l}
\text{if}
\begin{tikzcd}
id_P 
\arrow[r,mapsto,"\varphi_P"]
&
(\, P \morpharr{\alpha} A \,,\, P \morpharr{\beta} B\,)
\end{tikzcd}
\\ 
\text{then }
$\varphi \ = \ (\alpha\comp\phold,\beta\comp\phold)
 : \lscat C(\phold,P)\Longrightarrow K_{A,B}(-)
 : \lscat C^\op\to\catset$.
\end{tabular}
\end{equation}

\begin{exercise}
Prove~(\ref{Lecture9Exercise}) and further establish
\[
  \Efrac
    {\lscat C(\phold,P)\Longrightarrow K_{A,B}(\phold)}
    {K_{A,B}(P)}
  \text{nat.~in $X$}
\]
\end{exercise}

We now ask:  When is the natural transformation 
\[
(\alpha\comp\phold,\beta\comp\phold)
: \lscat C(\phold,P)\Longrightarrow\lscat C(\phold,A)\times\lscat C(\phold,B)
: \lscat C^\op\to\catset
\]
an isomorphism?
\begin{align*}
& \hspace*{-5.5mm}
(\alpha\comp\phold,\beta\comp\phold)
: \lscat C (\phold, P) 
    \natarrow \lscat C (\phold, A) \times \lscat C (\phold, B) 
\text{ is a natural isomorphism}
\\[1mm]
\text{iff }\ & 
\forall\, X\in\lscat C.\;
  (\alpha\comp\phold,\beta\comp\phold)_X
  : \lscat C (X, P) \morpharrow \lscat C (X, A) \times \lscat C (X, B)
  \textrm{ is a bijection}
\\[1mm]
\text{iff }\ & 
\forall\, X\in\lscat C.\;
  \forall\, f: X \to A\text{ in }\lscat C,\;
            g: X \to B\text{ in }\lscat C.\ 
    \exists!\, \langle f, g \rangle : X \to P\text{ in }\lscat C.\ 
      \alpha \comp \langle f, g \rangle = f 
      \,\wedge\,
      \beta \comp \langle f, g \rangle = g 
\\[1mm]
\textrm{iff }\ & 
A \overset{\alpha}{\longleftarrow} P \morpharr{\beta} B 
\text{ is a product diagram in }\lscat C
\end{align*}

By generalising the proof technique presented in
Subsection~\ref{subsec:proveiso}, %(using products and projection functors), 
we have seen that natural isomorphisms are a convenient way to describe
universal properties.

