\chapter{Natural isomorphisms}
\lecturedetails{2 November 2017}{M Fiore, A Ivašković, D Makwana }

%TODO: intro/clarification from start of the lecture?

We still focus our attention to certain kinds of natural transformations, and
are moving towards introducing \emph{representations}.

\section{Definition and basic properties}

\begin{definition}[Natural isomorphisms]
Given functors $F, G : \lscat A \to \lscat B$, a natural transformation
$\varphi: F \natto G$ is a \emph{natural isomorphism} whenever there exists a
(necessarily unique) functor $\varphi^{-1}: G \natto F$ such that:
\[ \varphi \comp \varphi^{-1} = \id{G} \]
\[ \varphi^{-1} \comp \varphi = \id{F} \]
\end{definition}

\begin{theorem}
For functors $F, G : \lscat A \to \lscat B$, a natural transformation
$\varphi: F \natto G$ is a natural isomorphism if and only if:
\[ \forall A \in \lscat A.\ \text{$\varphi_A: FA \to GA$ is an isomorphism} \]
\end{theorem}

\begin{proof}
We need to prove both implications of this statement.

\begin{itemize}

\item[($\Rightarrow$)] Let $\varphi$ be an isomorphism. So there exists a
$\varphi^{-1}$ that satisfies the properties from the definition:
\[ \varphi \comp \varphi^{-1} = \id{G} \]
Let $A \in \lscat A$. Then, since
\[ (\varphi \comp \varphi^{-1})_A = \varphi_A \comp \varphi_A^{-1} \]
\[ ( \id{G} )_A = \id{GA} \]
we can conclude:
\[ \varphi_A \comp \varphi_A^{-1} = \id{GA} \]

Similarly:
\[ \varphi_A^{-1} \comp \varphi_A = (\varphi^{-1} \comp \varphi)_A = (\id{F})_A
= \id{FA} \]

\item[($\Leftarrow$)] Suppose $\varphi_A: FA \to GA$ has an inverse
$\varphi_A^{-1}: GA \to FA$ for all $A \in \lscat A$. Then define:
\[ \varphi^{-1} \eqdef \setof{\varphi_A^{-1}: GA \to FA}_A \]
Then it is clear that:
\[ \varphi \comp \varphi^{-1} = \id{G} \]
\[ \varphi^{-1} \comp \varphi = \id{F} \]
It remains to show that the family $\varphi^{-1}$ is a natural transformation.
That is, we want to prove that the following diagram commutes:

\begin{center}
\begin{tikzcd}
    A \arrow{d}{\text{in $\lscat A$}}[swap]{\forall f} &
    GA \arrow{r}{\varphi_A^{-1}} \arrow{d}{Gf} & FA \arrow{d}{Ff} \\
    A & GA^{\prime} \arrow{r}{\varphi_{A^{\prime}}^{-1}} & FA^{\prime}
\end{tikzcd}
\end{center}

This is not difficult: since $\varphi$ is natural, the following diagram
commutes:
\begin{center}
\begin{tikzcd}
    FA \arrow{r}{\varphi_A} \arrow{d}{Ff} & GA \arrow{d}{Gf} \\
    FA^{\prime} \arrow{r}{\varphi_{A^{\prime}}} & GA^{\prime}
\end{tikzcd}
\end{center}
that is, $Gf \comp \varphi_A = \varphi_{A^{\prime}} \comp Ff$. Left composing
both sides of the equality with $\varphi_{A^{\prime}}^{-1}$ and right composing
them with $\varphi_{A}^{-1}$, we have
\[ \varphi_{A^{\prime}}^{-1} \comp Gf \comp \varphi_A \comp \varphi_A^{-1} =
\varphi_{A^{\prime}}^{-1} \comp \varphi_{A^{\prime}} \comp Ff
\comp \varphi_A^{-1} \]
which is just
\[ \varphi_{A^{\prime}}^{-1} \comp Gf = Ff \comp \varphi_A^{-1} \]
which is what we wanted to prove. Therefore $\varphi^{-1}$ is a natural
isomorphism.

\end{itemize}
\end{proof}

Intuitively, this means that, for a natural transformation to be an isomorphism,
it has to be an isomorphism at `every level' (`for every component').

\begin{exercise}
Investigate whether the analogous result holds for natural monomorphisms,
epimorphisms, sections and retractions (which are all defined similarly to
natural isomorphisms).
\end{exercise}

\section{A construction on functors}

\subsection{Characterising natural isomorphisms between restrictions}

Given a functor $F: \lscat A \times \lscat B \to \lscat C$ (so it is a functor
`of two variables'), define the functor $F (A, \phold): \lscat B \to \lscat C$
for a fixed $A$:

\begin{center}
\begin{tikzcd}
B \arrow[dd, "g"{name=L, right}] & &
F (A, B) \arrow[dd, "F(\id{A}{,} g)"{name=R, left}] \\
\arrow[rr, from=L, to=R, mapsto] & & \\
B^{\prime} &  & F (A, B^{\prime})
\end{tikzcd}
\end{center}

Intuitively, this means that we are performing a `restriction' of the arguments.

Similarly, we may for a fixed $B \in \lscat B$ define the functor
$F (\phold, B) : \lscat A \to \lscat C$ such that:

\begin{center}
\begin{tikzcd}
A \arrow[dd, "f"{name=L, right}] & &
    F (A, B) \arrow[dd, "F(f{,}\id{B})"{name=R, left}] \\
\arrow[rr, from=L, to=R, mapsto] & & \\
A^{\prime} &  & F (A^{\prime}, B)
\end{tikzcd}
\end{center}

We want to apply these to the hom functor:
\[ \lscat C (\phold, \Phold): {\lscat C}^\op \times \lscat C \to \catset \]
For a fixed $C \in \lscat C$, consider
$\lscat{C} (\phold, C): C^\op \to \catset$:

\begin{center}
\begin{tikzcd}
X & & \lscat C (X, C) \arrow[dd, ""{name=R, left}] & x \arrow[dd, mapsto] \\
& & & \\
Y \arrow[uu, "h"{name=L, right}] & & \lscat C (Y, C) & x \comp h
\arrow[rr, from=L, to=R, mapsto]
\end{tikzcd}
\end{center}

Now consider two fixed $A$ and $B$ in $\lscat C$. We are now interested in all
the natural transformations $\varphi$ of the form:
\[ \lscat C(\phold, A) \natarr{\varphi} \lscat C(\phold, B) :
{\lscat C}^\op \to \catset \]
Is it possible to characterise them precisely? The answer is \emph{yes} -- this
is because naturality is a very strong property.\footnote{Naturality is almost
like a function being polymorphic in a programming language. Think of the ML
or System F type construct $\forall \alpha.\ \alpha \to \alpha$: the only valid
expression of this type is the identity function.} It turns out out this is not
that difficult. Let $\varphi$ be the following family of functions:
\[ \varphi = \setof{\varphi_X: \lscat C (X, A) \to
\lscat C (X, B)}_{X \in \lscat C} \]
One of its members is $\varphi_A: \lscat C (A, A) \to \lscat C (A, B)$. Define:
\[ \widehat{\varphi} \eqdef \varphi_A (\id{A}) : A \to B \text{ in } \lscat C \]
We now claim that $\widehat{\varphi}$ completely determines all of $\varphi$
(that is $\varphi_X$, $\forall X$) \emph{because of the naturality condition}.

This is seen from the following commutative diagram, for any $X$:

\begin{center}
\begin{tikzcd}
    A & \lscat C (A, A) \arrow[rrr, "\varphi_A"] \arrow[ddd, "\phold \comp f"]
    & & & \lscat C (A, B) \arrow[ddd, "\phold \comp f"] & \\
    & & \id{A} \arrow[lu, phantom, "\rotatebox{135}{$\in$}"] \arrow[rrr, mapsto]
    \arrow[ddd, mapsto]
    & & & \widehat{\varphi} \arrow[lu, phantom, "\rotatebox{135}{$\in$}"]
    \arrow[ddd, mapsto] \\
    & & & & & \\
    X \arrow[uuu, "f"] & \lscat C (X, A) \arrow[rrr, "\varphi_X"]
    & & & \lscat C (X, B) & \\
    & & \id{A} \comp f & & & \widehat{\varphi} \comp f \\
    & X \morpharr{f} A \arrow[rrr, mapsto]
    \arrow[uu, phantom, "\rotatebox{90}{$\in$}"]
    \arrow[ru, phantom, "\rotatebox{45}{$=$}"] & & &
    \varphi_X (f) \arrow[ru, phantom, "\rotatebox{45}{$=$}"]
    \arrow[uu, phantom, "\rotatebox{90}{$\in$}"] &
\end{tikzcd}
\end{center}

All these natural transformations are characterised by being \textit{exactly}
those that are in bijective correspondence with maps
$\widehat{\varphi}: A \to B$. Here we have only shown one direction of this
claim.

\begin{exercise}
Show that every map $f: A \to B$ yields a natural transformation:
\[ \lscat C (\phold, A) \natarr{\widetilde{f}} \lscat C (\phold, B) \]
given by:
\[ \begin{array}{crcl}
    \widetilde{f_X}: & \lscat C (X, A) & \to & \lscat C (X, B) \\
    & x & \mapsto & f \comp x
\end{array} \]
\end{exercise}

\begin{exercise}
Show that these $\widehat{\phold}$ and $\widetilde{\phold}$ constructions are
inverses of each other:
\[ \varphi \mapsto \widehat{\varphi} \mapsto \widetilde{\widehat{\varphi}} =
\varphi \]
\[ f \mapsto \widetilde{f} \mapsto \widehat{\widetilde{f}} = f \]
\end{exercise}

\subsection{Technique for proving isomorphisms}\label{subsec:proveiso}

Note on notation: a double-lined fraction is sometimes used to express
bijective correspondence. The result proven above, is sometimes written as:

\[ \Efrac{\lscat C (\phold, A) \natarrow \lscat C (\phold, B) }{ \lscat C(A, B) }.\]

Suppose one was trying to show that two objects $A$ and $B$ are isomorphic.
This will usually involve constructing some isomorphism $ f : A \to B $.  In
practice, one can sometimes use the above result and instead come up with some
natural isomorphism $\lscat C (\phold, A) \natarr{\varphi} \lscat C (\phold,
B)$ (which amounts to showing $ \Efrac{X \to A}{X \to B} $ `is natural in $X$',
that is, holds for all $X \in \lscat C^\op $) and then define $f$ to be
$\widehat{\varphi}$.

\section{A product construction}

Fix a shape: $ A \overset{\alpha}{\longleftarrow} P \morpharr{\beta} B $. 
Define the functor $ K = \lscat C (\phold, A) \times \lscat C
(\phold, B)$ (the product of two functors is a a functor) and the infranatural
transformation $\varphi = \lscat C (\phold, P) \natarrow K $. Hence, for each $X
\in \lscat C^\op $, we have: 

\begin{center}
\begin{tikzcd}
    \lscat C(X, P) \arrow[rr] & & \lscat C (X, A) \times \lscat C (X, B) \\
    X \arrow[dd, "p", swap] & & X\arrow[dd, "\alpha \circ p", end anchor={[xshift=1em]north west}, swap] \arrow[dd, "\beta \circ p", end anchor ={[xshift=-1em]north east}] \\
    \arrow[rr, mapsto, "\varphi(p)", start anchor={[xshift=2em]}, end anchor={[xshift=-4em]}] & & \  \\
    P & & A \hskip 4em B
\end{tikzcd}
\end{center}

We now claim that $\varphi = (\alpha \circ \phold, \beta \circ \phold) = \setof{
{(\alpha \circ \phold, \beta \circ \phold)}_X }_{X \in \lscat C^\op}$ satisifies
the naturality condition.

% Dhruv: The diagram presented in lectures did not include the C(X,P)/C(Y,P) row
% or the C(-,P) arrows. It also said (alpha o g, beta o g), which I believe was
% incorrect, so wrote it as followed.

\begin{center}
\begin{tikzcd}
    X \arrow[rr, "{\lscat C (\phold, P)}"]
        && \lscat C(X, P) \arrow[rr, "{(\alpha \circ\;\phold,\;\beta \circ\;\phold)}_X"]
            \arrow[dd, "{C(g,P)}"]
        && K (X) \arrow[dd, "K(g)"]
        & (a,b) \arrow[dd, mapsto] \\
    & & & & & \\
    Y \arrow[uu, "g"] \arrow[rr, "{\lscat C (\phold, P)}"]
        && \lscat C (Y, P) \arrow[rr, swap, "{(\alpha \circ\;\phold,\;\beta \circ\;\phold)}_Y"]
        && K(Y)
        & (a \circ g, b \circ g)
\end{tikzcd}
\end{center}

We can determine what $\varphi$ looks like by calculating what $id_P$ is mapped
to:

\begin{center}
\begin{tikzcd}
    \varphi_P : & \lscat C (P, P) \arrow[rr, Rightarrow] && \lscat C (P, A) \times \lscat C (P, B)
\end{tikzcd}
\begin{tikzcd}
              & id_P \arrow[rr, mapsto] && (P \morpharr{\alpha} A,\quad P \morpharr{\beta} B)
\end{tikzcd}
\end{center}

\begin{exercise}

% Dhruv: The original phrasing was something along the lines of: "Can we
% reconstruct phi from A <-(a)- P -(b)-> B but I was unsure what that meant.

Prove $\varphi$ is natural. Finish the diagram such that it commutes, that is,
define $\lscat C (g,P)$ and $K(g)$ for $g : Y \to X$ in terms of $\alpha$ and
$\beta$ alone.
\end{exercise}

Having shown $\varphi$ is natural, we can then ask when $\varphi$ is a natural
isomorphism.

% Dhruv: There was an "exercise" here but I'm unsure what's left to do after
% the following was presented in the lecture.

\begin{align*}
    & \lscat C (\phold, P) \natarrow \lscat C (\phold, A) \times \lscat C (\phold, B)\\
    \textrm{iff  } & \forall X.\;\lscat C (X, P) \morpharrow \lscat C (X, A) \times \lscat C (X, B)
                        \textrm{ isomorphisms}\\
    \textrm{iff  } & \forall X.\;\forall f: X \to A,\;g : X \to B.\ 
                        \exists! \langle f, g \rangle : X \to P.\ 
                        \alpha \comp \langle f, g \rangle = f \wedge
                        \beta \comp \langle f, g \rangle = g \\
    \textrm{iff  } & A \overset{\alpha}{\longleftarrow} P \morpharr{\beta} B \textrm{ is a product.}
\end{align*}

By generalising the proof technique presented in
Subsection~\ref{subsec:proveiso} (using products and projection functors), we
have seen that a natural isomorphism is a convenient way to characterise a
universal mapping properties.
