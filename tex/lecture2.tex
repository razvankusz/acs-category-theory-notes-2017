\chapter{More universal problems}
\lecturedetails{10 October 2017}{M Fiore, A Ivašković, O Richardson }

The start of this lecture will provide some hints for the exercises from
Lecture 1.  We continue with a few more examples aimed at providing
mathematical background and building intuition.  Forming intuition is
important because there is no general method for finding universal initial and
final solutions.

\section{Hints To Exercises}

\subsection{Adding two sets}

Recall that in Exercise~\ref{ex:addsets}, we were looking for a universal
initial solution to adding sets $A$ and $B$:
\begin{center}
  \begin{tikzcd}
      {} & S \arrow[dd, dashed, "\exists! u"] & {} \\
      A \arrow[ru, "\alpha"] \arrow[rd, swap, "f"]
      & {}
      &
      B \arrow[lu, swap, "\beta"] \arrow[ld, "g"]
      \\
      {} & C & {}
  \end{tikzcd}
\end{center}
for all sets $C$, and functions $f$ and $g$.

When dealing with a problem we do not know immediately how to solve, it might be
useful to somehow `simplify' it in order to get some insight --- but not so much
that we lose all structural information.

Consider a special instance of this problem, where $B$ is a singleton set
$\setof{\ast}$:

\begin{center}
  \begin{tikzcd}
      {} & S & {} \\
      A \arrow[ru]
      & {}
      &
      \setof{\ast} \arrow[lu]
  \end{tikzcd}
\end{center}

Note that $\setof{\ast} \to S$ is mapping $\ast$ to some element in $S$, and
overall this represents the problem of adding an element to $A$
(Subsection~\ref{add-element-set}). We want to make sure that there is a unique
mapping to any other solution $C$, as illustrated below.

\begin{center}
\begin{tikzcd}
    {} & S \arrow[dd, dashed, "\exists!"] & {} \\
    A \arrow[ru] \arrow[rd]
    & {}
    &
    \setof{\ast} \arrow[lu] \arrow[ld]
    \\
    {} & C & {}
\end{tikzcd}
\end{center}

Here $S$ has to preserve the structure of $A$ and be augmented with an
additional element in some way. The image of elements of $A$ in $S$ have to be
distinct from the image of $\ast$, since otherwise there would not be a unique
way of representing the two in $C$. This is a map $a \mapsto [a]$ for $a \in A$.
Compare this to the previous problem, where we considered
$A \to A_\ast = \setof{\ast} \union \bigsetof{[x] \suchthat x \in A}$.

Taking this `tagging' idea further and generalising, we may consider the set:
\begin{equation*}
A \uplus B = \bigsetof{(0, a) \suchthat a \in A} \union
\bigsetof{(1, b) \suchthat b \in B}
\end{equation*}

We claim that $S$ should be $A \uplus B$, and that $\alpha: a \mapsto (0, a)$,
$\beta: b \mapsto (1, b)$ should be the maps for the initial solution.

\begin{exercise}
Show that this is indeed an initial solution.
\end{exercise}

\subsection{The dual problem}

In Exercise~\ref{ex:dual}, we were asked to find the dual solution to that
described previously:
\begin{center}
\begin{tikzcd}
    {} & S \arrow[ld] \arrow[rd] & {} \\
    A
    & {}
    &
    B
    \\
    {} & C \arrow[uu, dashed, "\exists!"] \arrow[lu, "f"] \arrow[ru, swap, "g"] & {}
\end{tikzcd}
\end{center}

This is a more involved problem than adding two sets, and simplifications might
not immediately reveal what needs to be done. Even so, to give hints, we will test
different structures for $B$ to gain some intuition.

First, consider $B$ to be a singleton set, $B = \nelem{1}$, where one
typically writes $\nelem{1}$ for a generic singleton set.  Then the problem
becomes:
\begin{center}
\begin{tikzcd}
    {} & S \arrow[ld] \arrow[rd] & {} \\
    A
    & {}
    &
    \nelem{1}
    \\
    {} & X \arrow[uu, dashed] \arrow[lu] \arrow[ru] & {}
\end{tikzcd}
\end{center}

There is only one function $X \to \nelem{1}$ and only one function $S \to
\nelem{1}$ (they both map everything to the only element of $\nelem 1$), so
the problem is reduced to:
\begin{center}
\begin{tikzcd}
    {} & S \arrow[ld] \\
    A
    & {} \\
    {} & X \arrow[uu, dashed] \arrow[lu]
\end{tikzcd}
\end{center}

and the final solution to this is clear:
$$
\begin{tikzcd}
    {} & A \arrow[ld, swap, "\id{A}"] \\
    A
    & {} \\
    {} & X \arrow[uu, dashed, "\exists! f"] \arrow[lu, "\forall f"]
\end{tikzcd}
$$

In general let's say $S$ is of the form $A \circledast B$ for some
construction $\circledast$, where for $B = \nelem{1}$ we can take
$A \circledast B$ to be $A$ (\ie~the structure matches that of of $A$).

What if $B = \emptyset$?
$$
\begin{tikzcd}
    {} & A \circledast \emptyset \arrow[ld] \arrow[rd] & {} \\
    A
    & {}
    &
    \emptyset
    \\
    {} & X \arrow[uu, dashed] \arrow[lu] \arrow[ru] & {}
\end{tikzcd}
$$
This is only possible and valid when $A \circledast \emptyset = \emptyset$
because there are no functions from a nonempty set into an empty set. Also
recall that there is only one function from $\emptyset$ to any other set --
namely the empty function (the function with the empty graph).

Now, what if $B = \setof{0, 1} = \nelem{2}$ --- namely, what should $A
\circledast \nelem{2}$ be? This step gives some insight into the general
solution.

\section{Isomorphism of initial solutions}

Recall that in our initial solution to the problem of generating a monoid
from a set, we actually found a number of different solutions, all of which
seemed somehow structurally similar:

\begin{enumerate}

\item $S \to \mathrm{List}(S)$, where $\mathrm{List}(S)$ is the monoid of lists
whose items have values drawn from $S$, with the empty list $[]$ being the
neutral element and the append operator \texttt{@} being the binary
operation.

\item $S \to S^\ast$, where $S^\ast$ is the set of all strings/words on $S$.
This is a monoid if the neutral element is $\epsilon$ (the empty word) and the
operation is string concatenation $\cdot$\,.

\item For $n \in \nats$ define $S_n$ in the following way:
\[
S_0 = \bigsetof{ \big(0, ()\big) }
\]
and for $n > 0$:
\[
S_n
= \bigsetof{\big(n, (s_1, \ldots, s_n)\big) \suchthat s_1, \ldots, s_n \in S}
\]

Then $\bigcup_{n\in\nats} S_n$ is a monoid with the operation $\cdot$ defined
as:
\[
\big(l, s_1, \ldots, s_l)\big)
\cdot
\big(k, (s_1^\prime, \ldots, s_k^\prime)\big)
=
\big(l + k, (s_1, \ldots, s_l, s_1^\prime, \ldots, s_k^\prime)\big)
\]
and with $\big(0, ()\big)$ being the neutral element.

\end{enumerate}

Each of these solutions are indeed `essentially the same' and this is a
fundamental property of universal solutions. This concept (isomorphism)
will be defined properly once we start dealing with category theory, but for
now we can still prove that two initial solutions of the kind we've been
dealing with are isomorphic, \ie~that they are in structural bijective
correspondence.

Given two initial solutions $M$~(Mine) and $Y$~(Yours), diagrammatically:
$$
\begin{tikzcd}
    S \arrow[r, "m"] \arrow[rd, swap, "y"]
    \arrow[rdd, swap, bend right = 20, "m"] &
    M \arrow[d, dashed, "\exists! u"] \\
    {} & Y \arrow[d, dashed, "\exists! v"] \\
    {} & M
\end{tikzcd}
$$

Note that we have also:
$$
\begin{tikzcd}
    S \arrow[r, "m"] \arrow[rd, swap, "m"] &
    M \arrow[d, dashed, "\exists! \id{M}"] \\
    {} & M
\end{tikzcd}
$$

Thus $v \comp u = \id{M}$. Analogously we conclude that $u \comp v = \id{Y}$
because:
$$
\begin{tikzcd}
    S \arrow[r, "y"] \arrow[rd, swap, "m"]
    \arrow[rdd, swap, bend right = 20, "y"] &
    Y \arrow[d, dashed, "\exists! v"]
    \arrow[dd, dashed, bend left = 80, "\exists! \id{Y}"] \\
    {} & M \arrow[d, dashed, "\exists! u"] \\
    {} & Y
\end{tikzcd}
$$

From this we conclude that \textit{any two initial solutions are isormorphic}:
$$
\begin{tikzcd}
    M \arrow[r, bend left = 20, "u"] \arrow[loop left, "\id{M}"] &
    Y \arrow[l, bend left = 20, "v"] \arrow[loop right, "\id{Y}"]
\end{tikzcd}
$$

From the perspective of computer science, one may think of universal problems
as implementation-independent specifications for which concrete solutions
provide actual implementations, all of which have to necessarily be
essentially the same --- structure should be preserved regardless of the
underlying implementation (it doesn't matter whether we're dealing with
strings, lists, \etc).

\begin{exercise}
From a set $S$ generate a \textit{commutative} monoid $M$, \ie~one that
satisfies $x \cdot y = y \cdot x$ for all $x, y \in M$ --- find the initial
solution.
\end{exercise}
\begin{proof}[Solution]
The intuition behind the solution is that the free commutative monoid on a set
is given by its \emph{set of finite multisets}.  Here is a formal
implementation of this idea.

Let $\approx$ be the binary relation on
$S^\ast \,\eqdef\, \bigcup_{\ell\in\nats}\setof{\ell} \times S^\ell$ given by
\begin{align*}
  & \big(i,(s_1,\ldots,s_i)\big)
  \approx
  \big(j,(t_1,\ldots,t_j)\big)
  \\
  \iff &
  \\
  &
  \text{ there exists a bijection }
  \sigma: \setof{1,\ldots,i}\to\setof{1,\ldots,j}
  \text{ such that } s_k = t_{\sigma(k)} \text{ for all } k = 1,\ldots,i
  \enspace.
\end{align*}

$\blacktriangleright$ Show that $\approx$ is an equivalence relation.

Define the set
\[
  \mathcal M_f(S) \ \eqdef \ S^\ast_{\,\raisebox{0mm}{$/\!\approx$}}
  \enspace.
\]

$\blacktriangleright$ Show that
\begin{align*}
  &\big(i,(s_1,\ldots,s_i)\big)
  \approx
  \big(i',(s'_1,\ldots,s'_{i'})\big)
  \text{ and }
  \big(j,(t_1,\ldots,t_j)\big)
  \approx
  \big(j',(t'_1,\ldots,t'_{j'})\big)
  \\
  &\text{imply }
  \big(i+j,(s_1,\ldots,s_i,t_1,\ldots,t_j)\big)
  \approx
  \big(i'+j',(s'_1,\ldots,s'_{i'},t'_1,\ldots,t'_{j'})\big)
  \enspace.
\end{align*}

We therefore have a binary operation
$\oplus: \mathcal M_f(S) \times \mathcal M_f(S) \to \mathcal M_f(S)$ given by
\[
  \big[\big(i,(s_1,\ldots,s_i)\big)\big]_\approx
  \oplus
  \big[\big(j,(t_1,\ldots,t_j)\big)\big]_\approx
  \ \eqdef \
  \big[\big(i+j,(s_1,\ldots,s_i,t_1,\ldots,t_j)\big)\big]_\approx
\]

$\blacktriangleright$ Prove that
\[
  \big(\
    \mathcal M_f(S)
    \ , \
    \bigsetof{\,\big( 0 , (\,)\big)\,}
    \ , \
    \oplus
    \ \big)
\]
is a commutative monoid.

We claim that the function
\[
  \iota: S \to \mathcal M_f(S) : s \mapsto \bigsetof{\,\big(1,(s)\big)\,}
\]
is an initial universal solution.

To this end, consider a commutative monoid $(M,\shortmid,\ast)$ and a function
$f:S\to M$.

$\blacktriangleright$ Prove that
\[
  \big(i,(s_1,\ldots,s_i)\big)
  \approx
  \big(i',(s'_1,\ldots,s'_{i'})\big)
  \implies
  \ast_i\big(f(s_1),\ldots,f(s_i)\big)
  =
  \ast_{i'}\big(f(s_1),\ldots,f(s'_{i'})\big)
\]
where $\ast_0(\,) \eqdef\ \shortmid$ and, for $n\in\nats$,
$\ast_{n+1}(x_1,\ldots,x_n,x_{n+1})\eqdef \ast_n(x_1,\ldots,x_n)\ast x_{n+1}$.

We therefore have a function $f^\#: \mathcal M_f(S)\to M$ given by
\[
  f^\#\big(i,(s_1,\ldots,s_i)\big)
  \eqdef
  \ast_i\big( f(s_1),\ldots,f(s_i)\big)
  \enspace.
\]

$\blacktriangleright$ Prove that $f^\#$ is a monoid homomorphism.

Assume that $h:\mathcal M_f(S)\to M$ is a monoid homomorphism such that
$h\comp\iota=f$.

$\blacktriangleright$ Prove that $h=f^\#$.
\end{proof}

\begin{exercise}
From a set $S$ generate a commutative and \textit{idempotent} monoid $M$ \ie~one
that also satisfies $x \cdot x = x$ for all $x \in M$ --- find the initial
solution.
\end{exercise}

\section{Orders and lattices}

For a set $P$ and $\leq\ \subseteq P \times P$ a binary relation, we will
consider the following properties.

\begin{definition}
Reflexivity, transitivity, antisymmetry
\begin{itemize}[noitemsep,topsep=0pt]
    \item $\leq$ is \textit{reflexive} if $\forall a \in P.\quad a \leq a$
    \item $\leq$ is \textit{transitive} if
      $\forall a,b,c \in P. \quad a \leq b \wedge b \leq c \Rightarrow a \leq c$
    \item $\leq$ is \textit{antisymmetric} if $\forall a,b \in P. \quad
      a \leq b \wedge b \leq a \Rightarrow a = b$
\end{itemize}
\end{definition}

There is standard terminology for binary relations subject to such axioms.

\begin{definition} Partial orders and preorders
\begin{itemize}[noitemsep,topsep=0pt]
\item $(P, \leq)$ is a \emph{preorder} (or \emph{quasiorder}) if $\leq$ is
reflexive and transitive
\item $(P, \leq)$ is a \emph{partial order} if is a preorder ($\leq$ is
  reflexive and transitive) and $\leq$ is also antisymmetric
\end{itemize}
\end{definition}

\begin{definition}
A partial order $(P, \leq)$ equipped with an associative, commutative, and
idempotent binary operation $\lor$, typically referred to as \emph{join} (or
\emph{sup}), that satisfies $x \leq y \iff x \vee y = x$ is a \emph{join (or
sup) semilattice}.

(It may also have a least (or bottom) element $\bot \in L$ satisfying
$\forall x \in L,~\bot\lor x = \bot$.)
\end{definition}

\subsection{Exercises}

We can now consider the problem of generating a join semilattice.  For a set
$S$, we want to freely construct a join
semilatice~$(\widetilde S,\vee_{\widetilde S})$ as follows:
% Create an example environment?
\begin{center}
    \begin{tikzcd}[ampersand replacement=\&]
        S \arrow[r, "\iota"] \arrow[dr, swap, "\forall\ f"] \&
        \tilde{S}
        \arrow[d, "\exists ! f^\#"]
        \& (\widetilde{S},\lor_{\widetilde{S}})
        \arrow[d, "\text{$f^\#$ a homomorphism}"]
        \& \text{a universal initial join semilatice}
        \\
        {}
        \& L \& (L,\lor_L) \& \text{any join semilatice}
    \end{tikzcd}\\[2mm]
    \text{where $h$ must preserve structure,
      \ie~$f^\#(x\lor_{\tilde{S}} y) = f^\#(x) \lor_L f^\#(y)$}
\end{center}
The exercise is to find the initial solution here, but we will start with some
intuition. The join semilatice generated from $S$ must contain each element
$x\in S$. Similarly, for any two, three, or, more generally, any finite
number of distinct elements of $S$, it must contain the join or union of all
of them.  Since the operator is commutative, the order doesn't matter, and
since it's idempotent, only one copy of each element of $S$ can exist. The
picture looks something like this:
\begin{align*}
&
&
x  & \quad\leftrightarrow\quad \setof{x} \subseteq S
\\
& x\not= y
& x \lor y & \quad\leftrightarrow\quad \setof{x,y} \subseteq S
\\
& x\not= y \not= z \not= x
&
x \lor y \lor z & \quad\leftrightarrow\quad \setof {x,y,z} \subseteq S
\\
& &
& \,\enspace\quad\vdots &
\\
&
x_i \not= x_j\ \forall i,j
& x_1 \lor \cdots \lor x_n & \quad\leftrightarrow\quad \setof{x_1, \ldots, x_n}
\subseteq S
\end{align*}

Thus, we might guess that the universal initial solution $\widetilde{S}$ is
the collection of finite subsets of $S$: $\mathcal{P}_{\mathrm{fin}}(S) =
\setof{X \suchthat
X \subseteq_\mathrm{fin} S}$, with the join operator $X \lor Y = X \cup Y$.
This is close, but not entirely correct.

\begin{exercise}
There is a `bug' in the above proposed solution. Find the actual initial
solution.\label{ex:bug}
\end{exercise}

