\chapter{Representability}
\lecturedetails{7 November 2017}{M Fiore, P Fernandes, P Bose}

\begin{definition}[Representation]
Consider a category $\mathcal{C}$ and a functor 
$K: \mathcal{C}^{op} \rightarrow \catset $.
A \emph{representation} of $K$ is given by:
\begin{itemize}
  \item an object $R \in \mathcal{C}$
  \item a natural isomorphism $\rho: \mathcal{C(\phold, R)}
    \overset\iso\natarrow K(\phold) $
  \end{itemize}
\end{definition}
Notice that representations are unique up to isomorphism since:
\[\mathcal{C}(\phold, R_1) \cong K(\phold) \cong 
\mathcal{C}(\phold, R_2)\ \iff R_1 \cong R_2\]

\begin{example}
If we consider $K_{A,B}(\phold) = \mathcal{C}(\phold, A) \times \mathcal{C}
(\phold, B)$, we have from last lecture that $A \times B$ coupled with the
natural isomorphism $(\pi_1 \circ \phold, \pi_2 \circ \phold)$ is a
representation.  
\end{example}

\begin{example}
Consider  $K: \mathcal{C}^{op} \rightarrow \catset $ such that 
$K(C) = \mathbb{1}$ for all $C\in\mathcal C$. Then we have that:
\[\mathcal{C}(\phold, R) \cong \mathbb{1}\]
So representing objects are terminal.

\end{example}
\section{Yoneda Lemma}
\begin{theorem}[Yonneda Lemma]
There is a natural bijective correspondence 
\[ 
  \Efrac
    {\mathcal{C}(\phold, R) \natarrow K}
    {K(R)}
\]
\end{theorem}
\begin{exercise}
Prove the Yonneda Lemma
\end{exercise}
\begin{example}
Consider the category:\\
\[\mathcal{W}
  \eqdef 
  (\, 0 \longleftarrow 1 \longleftarrow \cdots \longleftarrow i \longleftarrow
  i+1 \longleftarrow \cdots\,) 
  \qquad(i\in\nats)
\]
We then have that $K:\mathcal W^\op\to\catset$ must be of the form
\[K 
  \ : \ 
  (\, K(0) \longrightarrow K(1) \longrightarrow \cdots \longrightarrow
  K(i) \longrightarrow K(i+1) \longrightarrow \cdots \,) 
  \qquad(i\in\nats)
\] 
and that, for $n\in\mathcal W$, $\mathcal{W}(\phold,n):\mathcal
W^\op\to\catset$ must be of the form
\[
\mathcal{W}(\phold, n)
\ : \
\begin{tikzcd}
\mathcal{W}(0, n) \arrow[r] & \mathcal{W}(1, n) \arrow[r] & \cdots 
\arrow[r] & \mathcal{W}(n, n) \arrow[r] & 
\mathcal{W}(n+1,n) \arrow[r] & 
\cdots 
\\
\emptyset \arrow[u, phantom, "\rotatebox{90}{$=$}"] & \emptyset 
\arrow[u, phantom, "\rotatebox{90}{$=$}"] & \cdots & 
\nelem{1} \arrow[u, phantom, "\rotatebox{90}{$=$}"] & 
\nelem{1} \arrow[u, phantom, "\rotatebox{90}{$=$}"] & \cdots
\end{tikzcd}
\]
The natural transformations from $\mathcal{W}(\phold, n)$ to $K$ amount to 
commutative diagrams as follows:
\begin{center}
\begin{tikzcd}
\emptyset \arrow[r] \arrow[d] & \emptyset \arrow[r] \arrow[d] & \cdots
\arrow[r] & \emptyset \arrow[r] \arrow[d] & \nelem{1} \arrow[r] \arrow[d] &
\nelem{1} \arrow[r] \arrow[d] & \cdots 
\\
K(0) \arrow[r] & K(1) \arrow[r] & \cdots \arrow[r] & K(n-1) \arrow[r] & K(n)
\arrow[r] & K(n+1) \arrow[r] & \cdots 
\end{tikzcd}
\end{center}
We then see that to give a natural transformation 
$\mathcal{W}(\phold,n) \natarrow K$ is to give an element of $K(n)$. 
\end{example}

\begin{remark}
Recall the hom-functor 
$\mathcal{C}(\phold,\Phold): \mathcal{C}^{op} \times \mathcal{C} \rightarrow
\catset$.
\end{remark}

\begin{definition}[Yoneda Embedding]
By ``currying'' the hom-functor associated to a small category~$\mathbb C$, we
have the \emph{Yoneda embedding} 
$\yon: \mathbb{C} \rightarrow \catset^{\mathbb{C}^{op}}$ given for all arrows
$f: A \rightarrow B$ in $\mathbb{C}$ as:
\begin{center}
\begin{tikzcd}
A \arrow[mapsto]{r} \arrow{dd} & \mathbb{C}(-,A) \arrow{dd} \\
f \ \ \ \arrow[mapsto]{r} & \hspace*{12.5mm} \ \ (f \comp \phold) \\
B \arrow[mapsto]{r} & \mathbb{C}(-,B) \\
\end{tikzcd}
\end{center}

Recall that we have shown a bijective correspondence
\[ 
  \mathbb{C}(A,B) 
  \,\iso\,
  \text{Nat}\big( \mathbb{C}(-,A), \mathbb{C}(-,B)\big) 
\]
that is in fact given by
\[
  \yon_{A,B} 
  : \mathbb{C}(A,B) \to \catset^{\mathbb{C}^{op}}\big(\yon(A),\yon(B)\big)
\]
so that, $\yon: \mathbb C\to\catset^{\mathbb C^\op}$ intuitively provides an
isomorphic copy of $\mathbb C$ in $\catset^{\mathbb{C}^{op}}$ Such functors
are referred to as embeddings.
\end{definition}

\begin{definition}
Let $F : \mathcal{A} \rightarrow \mathcal{B}$ be a functor. We say $F$ is 
\emph{faithful} if 
$F_{X,Y} : \mathcal{A}(X,Y) \rightarrow \mathcal{B}\big(F(X),F(Y)\big)$ is an
injection for all objects $X,Y$ in $\mathcal{A}$.  We say that $F$ is
\emph{full} if $F_{X,Y}$ is surjective for all objects $X,Y$ in $\mathcal{A}$.
\end{definition}

\section{Representing hom-sets}

Can we represent/internalise the set of arrows $\mathcal{C}(A,B)$ of a
category $\mathcal C$ as an object, say $(A \Rightarrow B)$, in it?

\begin{definition}[Exponentials]
The \emph{exponential} of objects $A$ and $B$ in a cartesian category
$\mathcal{C}$ is a representation of the functor 
$\mathcal C(\phold\times A,B):\mathcal C^\op\to\catset$ of the parameterised
maps from $A$ to $B$; that is, it is given by an object $A\Rightarrow B$ in
$\mathcal C$ together with a natural isomorphism
\[  
  \mathcal{C}(\phold, A \Rightarrow B) 
  \cong 
  \mathcal{C}(\phold\times A, B)  
  \enspace.
\]
\end{definition}
%
Equivalently, such an exponential is given by an evaluation map 
\[
  \varepsilon_{A,B} \in \mathcal{C}\big((A \Rightarrow B) \times A, B\big)
\]
such that for all objects $X$ and arrows $f: X \times A \to B$ in
$\mathcal{C}$, there exists a unique arrow 
$\lambda^{A,B}(f) : X \rightarrow (A \Rightarrow B)$ making the following
diagram commute: 
\begin{center}
\begin{tikzcd}
(A \Rightarrow B) \times A \arrow[r, "\epsilon_{A,B}"] & B \\
X \times A \arrow[u, "\lambda^{A,B}(f) \times Id_A"] \arrow[ur,"f",swap] & {} 
\end{tikzcd}
\end{center}

\begin{exercise}
Suppose that $\mathcal{C}$ is a bicartesian (\ie~it has
finite products and finite coproducts) and closed (\ie~it has exponentials).
Then $\mathcal{C}$ is distributive; that is, the canonical map
\[  
  \delta: (A \times X) + (B \times X) \rightarrow (A + B )\times X
\]
is an isomorphism.
%
Build the inverse of $\delta$ above using that:
\[ 
  P \cong Q 
  \enspace \text{ iff }  \enspace 
  \Efrac{P \rightarrow Z}{Q \rightarrow Z}\ \text{natural in Z} 
\]
\end{exercise}
